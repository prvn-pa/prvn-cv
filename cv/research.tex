%-------------------------------------------------------------------------------
%	SECTION TITLE
%-------------------------------------------------------------------------------
\cvsection{Research}


%-------------------------------------------------------------------------------
%	CONTENT
%-------------------------------------------------------------------------------
\begin{cventries}
  \cventry
    {Post Doctoral Work} % Job title
    { } % Organization
    { } % Location
    {Jun. 2019 - Present} % Date(s)
    {\bf ORGANIC SEMICONDUCTORS FOR OLETs \& X-RAY SENSORS \vspace{0.5cm}}
      \begin{cvitems} % Description(s) of tasks/responsibilities
        \item[] {
\begin{spacing}{1.6}
{\bf Organic Light Emitting Transistors:}
\begin{itemize}
\item DFT analysis of biphenylyl end capped oligo-thiophenes with furan substitution
\item PVT growth of biphenyly/thiophene derivatives
\item Spin coating or thermal evaporation of dielectric layer / contacts
\item Basic characterizations such as SXRD, PXRD, PL, AFM, SEM
\item FET characterization using parametric analyser
\item Optical and Electrical pumping of OLETs
\end{itemize}

{\bf Organic X-Ray Sensors:}
\begin{itemize}
\item PVT growth of pentacene and tetracene systems
\item Thermal evaporation for thinfilms
\item Basic characterizations such as SXRD, PXRD, PL, AFM, SEM
\item Diode characterization using parametric analyser
\item Low power X-ray irradiation and corresponding I-V/t sampling
\end{itemize}

\end{spacing}}
      \end{cvitems}
%---------------------------------------------------------
  \cventry
    {Doctoral Work} % Job title
    { } % Organization
    { } % Location
    {Jan. 2013 - Jun. 2019} % Date(s)
    {{\bf METAL ORGANIC THIN FILMS FOR NLO APPLICATIONS \vspace{0.75cm}}
      \begin{cvitems} % Description(s) of tasks/responsibilities
        \item[] {
\begin{spacing}{1.6}
The core objective of the investigation is to analyse the effect of incorporation of metal ions in the benzimidazole (BMZ) medium and to analyse the potentiality of the synthesized system towards NLO applications. There are two strategies, (i) Computational analysis and (ii) Experimental evaluation, were primarily used for the analysis. Primarily, semiempirical quantum chemistry program MOPAC was used for the geometry optimization and molecular properties calculation. Parameters such as bond length, bond angle, dipole moment, energy gap, molecular energy and heat of formation were calculated and used for the interpretation of molecular polarizability and hyperpolarizability values. From the computational analysis three potential candidates Co(II), Cu(II) and Mn(II) were opted for the experimental studies. These metal-BMZ complexes were either deposited as thin films or casted as free standing films depending upon the associated substituent in the metal ion. These samples were subjected to structural and optical characterizations and evaluated for proto-types such as optical limiters (OL) and optical switches (OS). Since, benzimidazole complexes have anticancer activity and found to have a good thermo-optical behavior, they were also investigated for laser assisted anticancer activity.
\end{spacing}}
      \end{cvitems}}
\vspace{-1.5cm}
  \cventry
    {~~~~} % Job title
    {~~~~} % Organization
    {~~~~} % Location
    {~~~~} % Date(s)
    {{\bf Important achievements: \vspace{0.5cm}}
      \begin{cvitems} % Description(s) of tasks/responsibilities
	\setlength\itemsep{1.2em}
	\item[] ~~~\vspace{-0.5cm}
        \item Developed an improved version of chemical solution processing unit
        \item Physisorption based chemical deposition theory have been successfully developed
        \item Metal organic thin films of benzimidazole were deposited for the first time
	\item Home-made Z-scan, OL, OS setups were constructed
	\item Thermal-assisted anticancer activity with BMZ system was studied for the first time
      \end{cvitems}}


%---------------------------------------------------------

\end{cventries}
